%%%%% Change the four parameters in the line below:
% First is lecture #.
% Second is lecture title.
% Third is lecturer (either Anupam Gupta or Ryan O'Donnell).
% Fourth is your name.
\lecture{5}{Kernel, tournaments and king}{Anand Mishra}{Utkarsh Thusoo}





Continuation from Lecture 8

\section{Kernel in a directed graph}

Is a set of vertices $S \subseteq V(D)$ such that $S$ indicates no edges and every vertex outsied of S has a succssor in $S$. Concept of kernel is only defined for directed graph. 
% Here's a citation~\cite{Kar84a}.

\paragraph{Example 1:}Let us suppose we take a graph with 4 vertices and 4 edges which are in cycle. Kernel of this directed graph $ S_1 = \{1, 3\}$ which are not connected to each other. Similarly $ S_2 = \{2, 4\}$ is also a kernel, so there is no unique kernel in this case.
\myfig{.375}{graph_img1_BW.png}{Directed Graph to show kernels.}{fig:Directed Graph}

\paragraph{Example 2:}Let us suppose we take a graph with 4 vertices and 3 edges which are in cycle. Kernel of this directed graph $ S = \{2, 4\}$ which are not connected to each other. 
\myfig{.375}{graph_img2_BW.png}{Directed Graph to show kernels.}{fig:Directed Graph}

\paragraph{Example 3:}Let us suppose we take a graph with 6 vertices and 6 edges which are in cycle. Kernel of this directed graph $ S_1 = \{2, 4, 5\}$ , $ S_2 = \{1, 3, 6\}$ which are not connected to each other. 
\myfig{.375}{graph_img3_BW.png}{Directed Graph to show kernels.}{fig:Directed Graph}

\paragraph{Example 4:}For a complete graph kernel will be null. 
\myfig{.375}{graph_img4_BW.png}{Fully connected directed graph.}{fig:Directed Graph}

\paragraph{Example 5:}A graph with odd cycles will not have kernel. $ C_n$ where n is odd will not have kernel because we need to choose something in $ S $ to make a kernel and whatever you choose should be chosen a way such that they should be independent and successor of $ S $. In this case every node has only one successor hence if you dont choose any node then we have to choose its successor and simply put we can never have a scenario where kernel can be created.
\myfig{.375}{graph_img5_BW.png}{Directed graph}{fig:Directed Graph}

Let $ K $ be the kernel of $C_n$, then any $u,v \in K$ should have the following properties 

\begin{itemize}
    \item they should be independent;
    \item $ w \in K$ then $w$ must have successor in $K$
    \end{itemize}

\subsection{Outdegree and indegree}

Every node in directed graph will have a indegree and outdegree. Number of edges comming into and going out of the graph. 

In any directed graph sum of indegrees = sum of all degrees = number of edges in G i.e. \begin{equation}
    {\textstyle \sum_{v \in V(G)}} d^+(v) = {\textstyle \sum_{v \in V(G)}}  d^-(v) = {\textstyle \mid e(G) \mid}
 \end{equation}

\paragraph{Example 1:}For the given digram indegree and outdegree of each node would be\begin{itemize}
    \item $Outdegree$: For node $A$, $d^+(A) = 1$, For node $B$, $d^+(B) = 1$ 
    \item $Indegree$: For node $A$, $d^-(A) = 0$, For node $B$, $d^-(B) = 2$ 
    \end{itemize}
\myfig{.375}{graph_img6_BW.png}{Directed graph}{fig:Directed graph to find in/out degree}

\subsection{Orientation}

An orientation of graph $G$ is a directed graph $D$ obtained from $G$ by choosing an orientation $ x \to y$ or $y \to x$ for each edge $xy \in E(G) $. 

\myfig{.375}{graph_img9_BW.png}{Undirected to Directed graph}{fig:Directed graph to find in/out degree}
 
 
\section{Tournament in graph}

Tournament is an orientation of a complete graph. 

\subsection{King of the tournament}
King of the tournament is a vertex from where all the other vertices are reachable by a path of length of 2 at most.

\paragraph{Example 1:}Let us suppose we have a cricket tournament between different countries and graph should have directions. It can be assumed that $x \to y$ plays a match and $x$ wins against $y$. In the given example India is the king because all nodes can be reached from India.
\myfig{.375}{graph_img10_BW.png}{King of the tournament}{fig:Directed graph to find in/out degree}

\paragraph{Example 2:}A complete graph with 5 vertices and vertices to each other. All vertices are kings because they satisfy the definition of the king.
For node $B$
\begin{itemize}
    \item $B \to C$
    \item $B \to D \to A$
    \item $B \to D $
    \item $B \to D \to E$
    \end{itemize}
\myfig{.375}{graph_img11_BW.png}{King of the tournament}{fig:Directed graph to find in/out degree}

\paragraph{Every tournament has a king.}This can be proved with a contradiction. Suppose $u$ is a node with the highest outgoing degree. 
\begin{equation}
    {\textstyle N_o(u) +  N_I(u) + 1 = \mid V(G) \mid} 
 \end{equation} where $u$ is not a king.

Suppose $u$ is not the king. All the nodes is $N_I$ will be connected to all the nodes in $N_o$ and only thing we are not aware of is the direction.
\paragraph{Case 1:}If the direction is from $N_I \gets N_o$, then we can say that $u$ is king because we can reach from all the nodes of $N_I$ via just two edges $ u \to N_o \to N_I$. Hence every tournament must have a king.
\paragraph{Case 2:}If the firection is from $N_I \to N_o$. Lets suppose there is a node in $v$ in $N_I(u)$ and $v$ has outgoing edges in all the nodes i.e. $ v \to N_o$ and $ v \to u$. Number of outgoing edge of node $v$ = $1 + \mid N_o(u)\mid >$ number of outgong edge of $u$. This is a contradiction because we have assumed that $u$ is the node which has highest out degree. Hence this case is not applicable.

\myfig{.375}{graph_img12_BW.png}{King of the tournament}{fig:Directed graph to find in/out degree}


\begin{theorem} Prove or disprove if $D$ is a orientation of simple graph with 5 vertices then the vertices of $D$ cannot have distinct out degree.
\end{theorem}
\begin{proof}
We can create a graph where $x > y$ then $y \to x$ as below. In this way outegree of each of node \begin{itemize}
    \item $Node(1) = 4$ 
    \item $Node(2) = 3$ 
    \item $Node(3) = 2$
    \item $Node(4) = 1$
\end{itemize}
Hence the statement given in false because $D$ can have distinct outdegree.
\myfig{.375}{graph_img13_BW.png}{Example of simple graph}{fig:Directed graph to find in/out degree}
\end{proof}


\paragraph{Question}Give example of one real world realtion whose digraphs has no cycles.
\myfig{.375}{graph_img14_BW.png}{Real world example}{fig:Directed graph to find in/out degree}


\subsection{New d}

You might like to put use subsectioning these too.  An alternate way to put in a small subheading for a paragraph is to use the \begin{verbatim} \paragraph \end{verbatim} command.  For example:
 
\paragraph{A remembrance by Dantzig.}  The early days were full of intense excitement. Scientists, free at last from war-time pressures, entered the post-war period hungry for new areas of research. The computer came on the scene at just the right time. Economists and mathematicians were intrigued with the possibility that the fundamental problem of optimal allocation of scarce resources could be numerically solved. Not too long after my first meeting with Tucker there was a meeting of the Econometric Society in Wisconsin attended by well-known statisticians and mathematicians like Hotelling and von Neumann, and economists like Koopmans. I was a young unknown and I remember how frightened I was at the idea of presenting for the first time to such a distinguished audience, the concept of linear programming.


\section{Math stuff}

\begin{tikzpicture}
    \foreach[count=\i] \lseti/\lsetmi in {A/{$a$,$b$,$c$},B/{5,6,$z$}} {
        \begin{scope}[local bounding box=\lseti, x=2cm, y=0.5cm]
        \foreach[count=\j] \lj in \lsetmi {
            \node[minimum width=1em] (n-\j-\lseti) at (\i,-\j) {\lj};
        }
        \end{scope}
        \node[ellipse, draw, fit=(\lseti), 
        label={[name=l-\lseti]above:$\lseti$}] {};
    }
    \draw[->] (n-1-A) -- (n-1-B);
    \draw[->] (n-2-A) -- (n-2-B);
    \draw[->] (n-3-A) -- (n-3-B);
    \draw[->] (l-A) -- node[above]{$f$}(l-A.center-|l-B.west);
\end{tikzpicture} 

Please make an effort to typeset things nicely.  There are quite a few macros in the lpsdp.sty file.  Below are illustrated how to do some basic things; please study the \LaTeX\ carefully.

Here's a typical LP in standard/equational form, with an equation number on one of the constraints.
\begin{gather}
    \min \quad c^\top x                       \nonumber\\
    \begin{align}
        \text{s.t.} \quad   Ax &= b           \nonumber\\
                             x &\geq 0       \label{eqn:nonnegative}
    \end{align}
\end{gather}

\noindent Here's a reference to the~\eqref{eqn:nonnegative} nonnegativity constraint.  Some more LPs:

\begin{gather*}
    \min \quad 3x_1 - 5x_2 \\
    \begin{aligned}
        \text{s.t.} \quad   x_1 + 2x_2 &\leq 6\\
                            2x_1 + x_2 &\leq 6\\
                            2x_1 + 2x_2 &\geq 7\\
                            x_1,x_2 &\geq 0
    \end{aligned}
\end{gather*}

\begin{alignat*}{3}
    \text{minimize}&   \quad & 3x_1 - 5x_2 + 2x_3 - x_4&       & &\\
    \text{subject to}& \quad & x_1 + 2x_2 - 4x_3 + x_4 &\leq 6 & &\\
                           & & -x_1 + 3x_2 - x_3 - x_4 &\geq 7 & &\\
                           & & x_i &\geq 0  & &\quad \forall i = 1\dots 4
\end{alignat*}


Let's do some matrices:
\[
\begin{pmatrix}
    1 & \rho & \rho\\
    \rho & 1 & \rho\\
    \rho & \rho & 1\\
\end{pmatrix},
\quad \text{or alternately,} \quad
\begin{bmatrix}
    1 & 2 \\
    3 & 4 \\
\end{bmatrix}.
\]
More generically:
\[
    A = \begin{bmatrix}
            \vrule & \vrule & & \vrule\\
            A_{1} & A_{2} & \cdots & A_{n} \\
            \vrule & \vrule & & \vrule
        \end{bmatrix}
      = \begin{bmatrix}
            \text{---} & a_1 & \text{---} \\
            \text{---} & a_1 & \text{---} \\
                       & \vdots &  \\
            \text{---} & a_n & \text{---} 
        \end{bmatrix}
\]


Here's some more random typesetting: 
\begin{itemize}
\item ``$\PTIME$~vs.~$\NP$, where the former means time $\poly(n)$'';
\item $\wt{O}(f(x)) \text{ is } f(x) \cdot \polylog(f(x))$;
\item $\displaystyle 
        g(x) = \begin{cases}
                   \sin(2\theta) & \text{if $\theta \leq \pi$,}\\
                   \max\{\cos^2\theta, \tfrac13\} & \text{if $\theta > \pi$.}
               \end{cases}
      $
      
\end{itemize}
Please don't write $max(A)$ when you mean $\max(A)$, or $log(n)$ when you mean $\log(n)$, or "quotes" when you mean ``quotes''.\\

A theorem and a proof:
\begin{theorem} $(a+b)^2 = a^2 + 2ab + b^2$.
\end{theorem}
\begin{proof}
Let for the reader.
\end{proof}

\bigskip

Here's what to do if your proof ends on an equation:
\begin{proof}
It's easy:
\[
    (a+b)^2 = (a+b)(a+b) = (a+b)a + (a+b)b = a^2 + ba + ab + b^2 = a^2 + 2ab + b^2 \qedhere
\]
\end{proof}

Please insert figures liberally.  It's probably best if ``vector graphics'' are in pdf or png format, and ``bitmap graphics'' are in jpg format, but lots formats are supported.  There's a macro defined to make things easy.  Inkscape is a pretty reasonable, free program in which to draw figures.\footnote{Ryan: I admit, I sometimes draw figures in Powerpoint.}
 
%%%%%%%% FIGURES
% first parameter is a real number which is the scale factor; 
% second is the file name; 
% third is caption; 
% fourth gives the LaTeX label for future \ref
\myfig{.375}{example-figure.pdf}{The region $g_1 > 0, g_2 > -\tfrac{\rho}{\sqrt{1-\rho^2}}g_1$.}{fig:my-example}

Once you've inserted it, you can refer to it as Figure~\ref{fig:my-example}.\\

Finally, if you have citations, see the commented-out stuff in the \LaTeX~here.

%%%%%%%%%%% If you have citations then uncomment the line below:
%
%\insertbibliography{lpsdp}

